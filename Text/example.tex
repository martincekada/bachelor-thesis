\documentclass[times, utf8, zavrsni]{fer}
\usepackage{booktabs}
\usepackage[unicode]{hyperref}

\begin{document}

%inkscape je besplatan i dobar za crtanje, generirati samo svg, ali samo sliku, a ne cijlu stranicu!!
% TODO: Navedite broj rada.
\thesisnumber{000}

% TODO: Navedite naslov rada.
\title{Naslov}

% TODO: Navedite vaše ime i prezime.
\author{Arthur Dent}

\maketitle

% Ispis stranice s napomenom o umetanju izvornika rada. Uklonite naredbu \izvornik ako želite izbaciti tu stranicu.
\izvornik

% Dodavanje zahvale ili prazne stranice. Ako ne želite dodati zahvalu, naredbu ostavite radi prazne stranice.
\zahvala{}

\tableofcontents

\chapter{Uvod}
Uvod rada. Nakon uvoda dolaze poglavlja u kojima se obrađuje tema.

\begin{table}[h!]
\caption{Tablica ima caption iznad, a slika ispod!!}
\label{tbl:kvadrati}
  \centering
  \begin{tabular}{r | c c c}
  $x$ & 1 & 2 & 3 \\ \hline
  $f(x) = x^2$ & 1 & 4 & 9
  \end{tabular}
\end{table}

asdfsadfasdjfklasjdf;
sdfmlaskdf
asdfasdfja;sdlfasldjf

\begin{itemize}
  \item prvi redak
  \item sadkfasdlf
  \item sdfsdkf
  sdfasdf
  asdfasdf
  \item
\end{itemize}

\begin{enumerate}
  \item prvi redak
  \item sadkfasdlf
  \item sdfsdkf
  sdfasdf
  asdfasdf
  \item
\end{enumerate}

Kvadratna jednadzba jednoznacno je odredjena s tri koeficijenta: $a$, $b$ i $c$. I glasi: $a \cdot x^2_{i + 1} + b \cdots x_i + c$
Ili na drugi nacin: \[
  a \cdot x^2_{i + 1} + b \cdots x_i + c
\]
% sve ide s ref osim equationa, oni idu s eqref
i ovo je nastavak istog paragrafa. A rjesenje je dano i izrazom \eqref{eq:kvadjed} tekst iza izraza
A rjesenje jednadzbe glasi:
\begin{equation}
x_{1, 2} = \frac{-b \pm \sqrt{b^2 - 4ac}}{2}
\label{eq:kvadjed}
\end{equation}

Tabilca kvadrata borjevva priakzana je u nastavku \ref{tbl:kvadrati}


\begin{table}
\caption{Tablica ima caption iznad, a slika ispod!!}
\label{tbl:kvadrati2}
\centering
\begin{tabular}{r | c c c}
$x$ & 1 & 2 & 3 \\ \hline
$f(x) = x^2$ & 1 & 4 & 9
\end{tabular}
\end{table}


Evo jedne super knjige \citep{ungar2002uvod}

tekst prije slika
\begin{figure}[h!]
  \centering
  \includegraphics[width=0.75\linewidth]{asdf.png}
  \caption{asdfasdfasdf}
  \label{fig:slika}
\end{figure}


voce koje jako volime je:
\begin{itemize}
  \item jabuke,
  \item kruske i
  \item sljive.
\end{itemize}

nesto je nalaseno \emph{ovaj je jos \emph{dodatno} naglaseno}


google je (\url{http://google.com}) kul tranz


\chapter{Zaključak}
Zaključak.

\bibliography{literatura}
\bibliographystyle{fer}

\begin{sazetak}
Sažetak na hrvatskom jeziku.

\kljucnerijeci{Ključne riječi, odvojene zarezima.}
\end{sazetak}

% TODO: Navedite naslov na engleskom jeziku.
\engtitle{Title}
\begin{abstract}
Abstract.

\keywords{Keywords.}
\end{abstract}

\end{document}
