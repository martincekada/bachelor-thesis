Konstruktivni optimizacijski algoritmi su široko korišteni algoritmi za heurističku optimizaciju. Inačica konstruktivnog optimizacijskog algoritma
koji je korišten u ovom radu se temelji na radu (link na paper).

1. Opći matematički model algoritma
Promatramo problem s konačnim skupom izvodljivih rješenja S i funkcijom kazne f : S → R. Neka je skup S* skup optimalnih rješenja te isključimo trivijalna rješenja,
odnosno pretpostavim da |S| > 1 i da S∗ ≠ S. Neka se svako rješenje iz skupa S može zapisati kao konačan niz simbola s = (s1,...,sL) iz skupa abecede A : =
{a1,..., aK}. L je fiksna duljina rješenja. Model je generaliziran funkcijom izvodljivosti Ci(y, a) koja dodjeljuje težinu svakom a ∈ A za svako moguće parcijalno
rješenje y duljine i. Prema tome, ako je Ci(y, a) = 0, parcijalno rješenje y se ne može nastaviti tako da se na njega nadoda znak a. Što je vrijednost funkcije Ci(y, a) veća,
toliko je poželjnije da se niz nastavi znakom _a_ pohlepno (engl. greedy) gledano. Pretpostavlja se da su vrijednosti funkcije normirane. Model gradi rješenja korak po korak
dodavajući nove simbole na desni kraj niza sve dok rješenje nije potpuno. Formalnije rečeno, definiramo skup Ri izvodljivih parcijalnih rješenja rekurzivno kako slijedi:
neka _dijamant_ pretstavlja prazan string. Pretpostavljamo da je dano:
C0(_dijamant_, a) :A → [0, 1], Suma C0(_dijamant_, a) = 1 (za svaki a elemnt A)
što pretstavlja poželjnost i izvodljivost znaka _a_ na prvom mjestu rješenja. Nadalje definiramo:
R1 : = {a ∈ A | C0(, a) > 0}
Kao skup izvodljivih rješenja duljine 1. Pretpostavimo da je definiran skup Ri za neki i  ∈ {0,..., L − 1} i da je dano:
Ci(y, ·) :A → [0, 1], Suma-za-svaki-aEa(Ci(y, a) = 1) za svaki y ∈ Ri.
Neka (y, a) označava konkateniranje simabola a ∈ A na desni kraj parcijalnog rješenja y. Definiramo:
Ri+1 : = { (y, a) | y ∈ Ri, a ∈ A, Ci(y, a) > 0 }
i neka je S := RL. Za svaki y ∈ Ri, i ∈ {0,..., L − 1}, neka je
Ci(y) : = {a ∈ A | Ci(y, a) > 0}
bude potpora Ci(y, ·). -- We use the abbreviation I : = {0,..., L − 1} in the sequel. --

Ovakva generalizacija modela ne postavlja gotovo nikakva ograničenja na optimizacijski problem. Odnosno, odgovarajućim odabirom A, S i C(y, a) svaki se problem može
prilagoditi na ovaj model, no efikasnost može varirati. Model je razumnije koristiti u slučajevima kada je |A| << |S|. Ovo dodatno potvrđuje
teorem s kraja prethodnog poglavlja koji tvrdi da svaki algoritam ne daje jednako dobre rezultate na svakom problemu.


2. Generalizirani algoritam
U suštini, algoritam razvija distribuciju nad skupom S = RL svih izvodljivih rješenja dajući pritom veliku vjerojatnost optimalnim rješenjima iz skupa S*.
Neka P(A) označeva skup svih vjerojatnosti nad skupom A. Tada je p ∈ P(A)^L, p = (p(1), . . . , p(L)) vjerojatnost nad skupom A^L koja opisuje odabiranje rješenja
s = (s1,...,sL) ∈ A^L pri čemu je L simbola s1,...,sL odabrano neovisno. Pritom je p(i) = p(a;i)a∈A ∈ P(A) je distribucija vjerojatnosti za simbol na i-toj lokaciji.
Ulazni podaci za algoritam su:
1. Funkcija poželjnosti Ci(·, ·), i ∈ {0,..., L − 1}
2. Niz koeficijenata izglađivanja (Qt)t≥1 uz Qt ∈ (0, 1);
3. Veličinu uzorka N i veličinu poduzorka Nb
4. Početnu distribuciju p0 ∈ P(A)^L

Početak:
Za t = 0, postavi p = p0. Iteriraj kroz korake t = 1, 2,... sve dok uvijet zaustavljanja nije zadovoljen.

Uzorkovanje (engl. Sampling):
Ako je trenutna distribucija p ∈ P(A)^L, tada je vjerojatnost uzrokovanje rješenje s = (s1,...,sL) ∈ S dana izrazom:
Qp(s) : = Qp(s1; 1, )·Li=2 Qp(si;i, (s1,...,si−1))
pri čemu je:
Qp(a;i, y) : = p(a, i)Ci−1(y, a)a	∈A p(a, i)Ci−1(y, a
vjerojatnost da se izvodljivi simbol a ∈ A nadoda na poziciju i izvodljivog parcijalnog rješenja y ∈ Ri - 1. Koristi se konvencija 0/0 = 0.
Na ovakav način, algoritam uzorkuje N rješenja s(1),...,s(N) neovisno i podjednako distribuirano.


Ocjenjivanje (engl. Evaluation):
Neka su uzorci x : = (s(1),...,s(N)) poredani prema funkciji kazne f:
f(s1) < f(s2) ...
te neka je odabrano najboljih Nb uzoraka Nb : = {s(n1), s(n2),...,s(nNb)}. Nakon toga određujemo relativnu frekvenciju simbola a na poziciji i ∈ {1,..., L}
u odabranom djelu uzorka
w(a;i, x) : = 1/Nb s∈Nb 1{a}(si) (4)
i prikupimo te frekvencije za svaki a ∈ A te kreiramo w(i, x) : = w(a;i, x) a∈A i
w(x) : = w(1, x), . . . ,w(L, x).
Tada je w(x) distribucija vjerojatnosti za P(A)^L koja daje relativne frekvencije sinbola iz boljeg dijela uzorka x uzorkovanog s vjerojatnosti Qp.



Ažuriranje:
Trenutnu distribuciju p ažuriramo kao kombinaciju p i relativne frekvencije w(x)
p = (1 - Qt+1) * p + Qt+1 * w(x)
U idućem koraku se brojač t uvećava za 1 i korak uzorkovanja se obavlja s novom ditribucijom p.






2. Prilagodba modela -> ovo mozda prebaciti u poglavlje 4
mapiranje opceg na nas konkretan problem




usporedba s mravljim algoritmom