Problemi raspoređivanja su specijalizacija transportnih problema te su jedni od temeljnih optimiziacijskih problema.
Općenito gledano, problemi raspoređivanja bi se mogli definirati na sljedeći način:
Problem se sastoji od agenata i od zadataka. Svakom agentu može biti dodjeljen bilo koji od zadataka uz određenu cijenu, a cijena
može varirati ovisno o uparivanju agenta i zadatka. Potrebno je svakom agentu dodijeliti zadadatak tako da ukupna cijena dodjeljivanja
bude minimalna.

Ovakvi problemi bi se mogli pokušati rješiti na način da se generira svaka kombinacija dodjeljivanja agenata i zadataka te da se
odredi dodjeljivanje s najmanjom cijenom. Ukoliko bi se dodjeljivanje vršilo za _n_ agenata i _n_ zadataka, složenost dodjeljivanja bi bila
_n!_. Porastom broja _n_ vrijeme potrebno da se na ovaj način odredi dodjeljivanje s najmanjom cijenom vrlo brzo postaje preveliko da bi se
izračunalo u realnom vremenu. Zbog toga kod problema raspoređivanja (i općenito optimizacijskih problema) ne tražimo iscrpno optimlano rješenje,
nego različitim pristupima pokušavamo pronaći dovoljno dobro rješenje. U nastavku slijedi opis nekih specifičnih problema raspoređivanja.

1. Primjer 1 (http://java.zemris.fer.hr/nastava/pioa/knjiga-0.1.2013-12-30.pdf)
Zadana je funkcija g(x, y, z) nad domenom [−300, 500] × [−300, 500] × [−300, 500] ⊂
R × R × R. Pronaći točku (x, y, z) za koju funkcija g poprima maksimalnu vrijednost.

2. Izrada satnice rasporeda
Satničaru su na raspolaganju popis kolegija koji se predaju, popis studenata i njihov izbor kolegija, popis slobodnih dvorana i termina,
željeni tjedni broj predavanja za svaki od kolegija te popis nastavnika koji predaju određene kolegije. Satničar treba zadovoljiti
slijedeće uvijete:
  - Svi studenti imaju zakazana sva predavanja i mogu ih slušati bez kolizija
  - Niti jedan nastavnik ne drži više predavanja istovremeno
  - Niti ujednu dvoranu ne smije biti smješteno više studenata nego što je kapacitet dvorane
  - Niti ujednu prostoriju ne smiju biti smještena dva predavanja istovremeno

Također, bilo bi poželjno kada bi satničar osigurao da:
  - Student u danu ima barem dva predavanja ili niti jedno
  - Student ima minimalan broj rupa u danu
  - Nastavnik ima minimalan broj rupa u danu
  - Nastavnik ima minimalan broj promjena dvorana u danu



3. Izrada rasporeda međuispita (http://ferko.fer.hr/ferko/EPortfolio!dlFile.action?id=78)
Primjer u nastavku je pojednostavljen i prilagođen organizaciji međuispita na Fakultetu Elektrotehnike i računarstva.
Prilikom izrade ispita dostupni su podaci o predmetima za koje treba održati ispit, podaci o slobodnim terminima te
podaci o studentima i ispitima kojima oni mogu pristupiti. Poterbno je izraditi raspored međuispita tako da niti jedan
student ne piše istovremeno dva ispita, da svaki student može pristupiti svakom svojem ispitu
te tako da se niti u jednoj dvorani ne pišu istovremeno dva ispita. Bilo bi poželjno
da student ima što ravnomjernije raspoređen broj slobodnih dana između ispita.




Na prethodnim primjerima se može primjetiti da se neki od uvjeta moraju ispuniti, dok su neka samo poželjna. To su tvrda
(engl. hard) i meka (engl. soft) ograničenja (engl. constraints). Tvrda ograničenja moraju biti ispunjena kako bi rješenje bilo
prihvatljivo (npr. u primjeru 2 ne može konačno rješenje biti ono u kojem nastavnik istovremeno predaje u dva termina). S druge strane,
meka ograničenja nisu obavezna, ali što su ona ispunjenija, to je rješenje bolje.