\documentclass{beamer}
\usetheme{Madrid}
\usecolortheme{beaver}
\usepackage[utf8]{inputenc}
\usepackage[croatian]{babel}
\usepackage[T1]{fontenc}
\usepackage{appendixnumberbeamer}

\newcommand{\floor}[1]{\lfloor #1 \rfloor}

%Information to be included in the title page:
\title[Obrana završnog rada] %optional
{Primjena konstruktivnog optimizacijskog algoritma na problem
rasporeda studenata}

\subtitle{Završni rad br. 6297}

\author[Martin Čekada] % (optional, for multiple authors)
{Martin Čekada}

\institute[FER] % (optional)
{
  Sveučilište u Zagrebu

  Fakultet elektrotehnike i računarstva
}

\date[Zagreb, Srpanj 2019] % (optional)
{Zagreb, Srpanj 2019}

\logo{\includegraphics[height=1cm]{FER_logo.pdf}}

\begin{document}

\frame{\titlepage}

\begin{frame}
\frametitle{Sadržaj}
\tableofcontents
\end{frame}

\section{Opis algoritma}
\begin{frame}
\frametitle{Opis algoritma}
\begin{itemize}
  \item Konstruktivni algoritam
  \item Optimizacijski problem
  \item Temeljen na radu \textit{Asymptotic Properties of a Generalized Cross-Entropy Optimization Algorithm} (Wu, Kolonko; 2014)
  \end{itemize}
\end{frame}


\begin{frame}
\frametitle{Primjer}
  \begin{table}
    % \caption{Početna distribucija}
    \centering
    \begin{tabular}{c | c | c | c }
       & Student 1 & Student 2 & Student 3  \\ \hline
      T1 & 0.5 & 0.5 & 0.5 \\ \hline
      T2 & 0.5 & 0.5 & 0.5
    \end{tabular}
  \end{table}


  Generirana rješenja: (T1, T1, T2); (T2, T1, T2); (T2, T2, T2)


  \begin{table}
    \centering
    % \caption{Uzorkovana distribucija}
    \begin{tabular}{c | c | c | c }
       & Student 1 & Student 2 & Student 3  \\ \hline
      T1 & 0.33 & 0.67 & 0 \\ \hline
      T2 & 0.67 & 0.33 & 1
    \end{tabular}
  \end{table}

  Nova distribucija:
  \begin{table}
    % \caption{Nova distribucija}
    \centering
    \begin{tabular}{c | c | c | c }
       & Student 1 & Student 2 & Student 3  \\ \hline
      T1 & 0.415 & 0.585 & 0.75 \\ \hline
      T2 & 0.585 & 0.415 & 0.25
    \end{tabular}
  \end{table}
\end{frame}


\section{Opis problema}
\begin{frame}
\frametitle{Opis problema}
\begin{itemize}
  \item Skup studenata
  \item Skup termina
  \item Funkcija dodjeljivanja kazne
  \begin{itemize}
    \item Tvrda ograničenja
    \item Meka ograničenja
  \end{itemize}
\end{itemize}
\end{frame}

\section{Nadogradnje algoritma}

\subsection{Inicijalna distribucija}
\begin{frame}
\frametitle{Inicijalna distribucija}
\begin{itemize}
  \item Želja: spriječiti nastajanje kolizija
  \item Pri inicijalizaciji rada algoritma je poznat skup studenata i skup termina
  \item Umjesto uniformne distribucije, smanjiti vjerojatnost pridjeljivanja u kojima nastaju kolizije
\end{itemize}
\end{frame}

\subsection{Heuristike}
\begin{frame}
\frametitle{Heuristike}
\begin{itemize}
  \item Rebalansiranje trenutne distribucije
  \item Redistribucija vjerojatnosti prepunjenih termina
\end{itemize}
\end{frame}

\subsection{Prioritetni red}
\begin{frame}
\frametitle{Prioritetni red}
\begin{itemize}
  \item Želja: smanjiti specijalizaciju algoritma
  \item Koeficijent i veličina
  \item Periodičko pražnjenje reda
\end{itemize}
\end{frame}

\section{Rezultati}
\begin{frame}
\frametitle{Rezultati}

\begin{table}
  \caption{Imenovanje problema}
  \label{tbl:opis_problema}
  \centering
  \resizebox{\textwidth}{!}{
  \begin{tabular}{c | c }
    $Ime$ & $Opis$    \\ \hline
    Problem 1 & Laboratorijske vježbe iz Digitalne logike 2018./2019. \\ \hline
    Problem 2 & Laboratorijske vježbe iz Digitalne logike 2018./2019. \\ \hline
    Problem 3 & Laboratorijske vježbe iz Digitalne logike 2018./2019. \\ \hline
    Problem 4 & Laboratorijske vježbe iz Digitalne logike 2018./2019. \\ \hline
    Problem 5 & Laboratorijske vježbe iz Objektno orijentiranog programiranje 2018./ 2019. \\ \hline
    Problem 6 & Laboratorijske vježbe iz Objektno orijentiranog programiranje 2018./ 2019.
  \end{tabular}
  }
\end{table}

\begin{table}
  \caption{Podaci o problemima}
  \label{tbl:podaci_problema}
  \centering
  \resizebox{\textwidth}{!}{
  \begin{tabular}{c | c | c | c | c | c | c  }
    $Broj$     &  $Problem 1$ & $Problem 2$ & $Problem 3$ & $Problem 4$ & $Problem 5$ & $Problem 6$ \\ \hline
    Studenata  &  371       & 372       & 372       & 372       & 515       & 515       \\ \hline
    Termina    &  27        & 27        & 27        & 27        & 32        & 32        \\ \hline
    Maksimalno &  14        & 14        & 14        & 14        & 19        & 19        \\ \hline
    Minimalno  &  0         & 0         & 0         & 0         & 17        & 17
  \end{tabular}
  }
\end{table}
\end{frame}

\begin{frame}
  \frametitle{Rezultati rasporeda za kolegij Digitalne logike}
  \begin{itemize}
    \item Veličina uzorka = 300
    \item Broj najboljih rješenja = 50
    \item Duljina prioritetnog reda = 50
    \item Koeficijent prioritetnog reda = 0.3
    \item Koeficijent zaglađivanja = 0.6
  \end{itemize}

  \begin{table}
    \caption{Rezultati rasporeda za kolegij Digitalne logike}
    \label{tbl:diglog_300}
    \centering
    \resizebox{\textwidth}{!}{
      \begin{tabular}{c | c | c }
        Ime problema &  Broj studenata kojima je pridružen optimalan termin & Broj studenata za koje postoji optimalan termin  \\ \hline
      $Problem1$ & 73 & 207 \\ \hline
      $Problem2$ & 61 & 210 \\ \hline
      $Problem3$ & 59 & 204 \\ \hline
      $Problem4$ & 63 & 216
    \end{tabular}
    }
  \end{table}
\end{frame}



\begin{frame}
  \frametitle{Zaključak i daljnji rad}
  \begin{itemize}
    \item Algoritam je uspješno zadovoljio tvrda ograničenja za sve primjere
    \item Paraleliziranje postupka uzorkovanja i ocjenjivanja
    \item Redistribucija vjerojatnosti prepunjenih termina
  \end{itemize}
\end{frame}

\begin{frame}
  \frametitle{Kraj}
  \centering
  \large
    Hvala na pažnji!
\end{frame}












\appendix

\section{Primjena na problem vodećih jedinica}
\begin{frame}
\frametitle{Primjena na problem vodećih jedinica}
\begin{block}{Problem vodećih jedinica (engl. \textit{LeadingOne problem})}
  Problem se sastoji od generiranja niza nula i jedinica s ciljem maksimiziranja broja početnih jedinica. Optimalno rješenje je ono u kojem se niz sastoji isključivo od jedinica.
  \end{block}

  \begin{block}{Teorem}
    Uz odabir konstantnog parametra izglađivanja $\varrho_t = \varrho$, veličine uzorka $N = L^{(2 + \epsilon)}$, uz $\epsilon > 0$
    i $N_b =  \floor{(\beta N)}$ za
    $0 < \beta < \frac{1}{3e} \prod^{\infty}_{m=1} (1 - (1 - \varrho)^m)$. Uz početnu distribuciju
    $\prod_0(1, i) \equiv \frac{1}{2}$, odnosno jednoliku distribuciju, za prethodno definirani
  problem vodeće jedinice vrijedi $\mathbb{P}(\tau < L) \rightarrow 1$ kada $L \rightarrow \infty$.
  Pri čemu je $\tau := min \{ t \geq 0 | X_t \cap S^* \neq \emptyset \}$
    \end{block}
\end{frame}

\begin{frame}[t]
\frametitle{Vrednovanje}
\begin{columns}[t]
  \column{0.5\textwidth}
  \begin{table}[]
    \caption{Utjecaj duljine uzorka $L$ na broj iteracija ($\epsilon = 0.5$, $\beta = 0.09$)}
    \label{tbl:promjenaL}
    \centering
    \begin{tabular}{c | c | c | c | c}
      $L$ & $N$ & $N_b$ & $\varrho$ & $i$ \\ \hline
      10 & 316 & 28 & 0.8 & 6 \\ \hline
      20 & 1788 & 160 & 0.8 & 9 \\ \hline
      30 & 4929 & 443 & 0.8 & 13 \\ \hline
      40 & 10119 & 910 & 0.8 & 16 \\ \hline
      50 & 17677 & 1590 & 0.8 & 20 \\ \hline
      60 & 27885 & 2509 & 0.8 & 23 \\ \hline
      70 & 40996 & 3689 & 0.8 & 26 \\ \hline
      80 & 57243 & 5151 & 0.8 & 30 \\ \hline
      90 & 76843 & 6915 & 0.8 & 33 \\ \hline
      100 & 100000 & 9000 & 0.8 & 37
    \end{tabular}
  \end{table}
  \column{0.5\textwidth}
  \begin{table}[]
    \caption{Utjecaj parametra izglađivanja $\varrho_t$ na broj iteracija ($\epsilon = 0.5$, $\beta = 0.09$)}
    \label{tbl:promjenaQ}
    \centering
    \begin{tabular}{c | c | c | c | c}
      $L$ & $N$ & $N_b$ & $\varrho$ & $i$ \\ \hline
      50 & 17677 & 6 & 0.2 & 43 \\ \hline
      50 & 17677 & 82 & 0.3 & 33 \\ \hline
      50 & 17677 & 279 & 0.4 & 29 \\ \hline
      50 & 17677 & 563 & 0.5 & 25 \\ \hline
      50 & 17677 & 881 & 0.6 & 23 \\ \hline
      50 & 17677 & 1195 & 0.7 & 21 \\ \hline
      50 & 17677 & 1483 & 0.8 & 19 \\ \hline
      50 & 17677 & 1736 & 0.9 & 18 \\ \hline
      50 & 17677 & 1950 & 1.0 & 15
    \end{tabular}
  \end{table}
  \end{columns}

\end{frame}

\begin{frame}
  \frametitle{Rezultati rasporeda za kolegij Objektno orijentiranog programiranja}
  \begin{itemize}
    \item Veličina uzorka = 300
    \item Broj najboljih rješenja = 50
    \item Duljina prioritetnog reda = 50
    \item Koeficijent prioritetnog reda = 0.3
    \item Koeficijent zaglađivanja = 0.6
  \end{itemize}
  \begin{table}
    \caption{Rezultati rasporeda za kolegij Objektno orijentiranog programiranja}
    \label{tbl:oop_300}
    \centering
    \resizebox{\textwidth}{!}{
    \begin{tabular}{c | c | c }
      Ime problema &  Broj studenata kojima je pridružen optimalan termin & Broj studenata za koje postoji optimalan termin  \\ \hline
      $Problem5$ & 77 & 278 \\ \hline
      $Problem6$ & 76 & 219
    \end{tabular}
    }
  \end{table}
\end{frame}

\begin{frame}
  \frametitle{Funkcija dodjeljivanja kazne}
  \begin{itemize}
    \item Za svaku koliziju: $8000$
    \item Za svaki prepunjeni (analogno i potpunjeni) termin: $5000 + koeficijentPrepunjenja \cdot (popunjenost - maksimalanBrojStudenataZaTermin)^2$
    \item Za svako produljivanje trajanja dana studentu: $50$
    \item Za svako pridjeljivanje koje je studentu na slobodan dan: $50$
  \end{itemize}
  \end{frame}

  \begin{frame}
  \frametitle{Utjecaj nadogradnji na rad algoritma}
  \begin{table}
    \caption{Utjecaj nadogradnji algoritma}
    \label{tbl:razvoj}
    \centering
    \resizebox{\textwidth}{!}{
    \begin{tabular}{c | c | c | c | c | c | c | c | c | c  }
      Inicijalna dist. & Prior. red & Rebalansiranje & Redistribuiranje & Prepunjenost & Kolizije & Medijan & Srednja vrijednost & Najmanji & Najveći \\ \hline
      $\bot$ & $\bot$ & $\bot$ & $\bot$ & 0 & 2 & 346000 & 425990 & 346000 & 676700 \\ \hline
      $\top$ & $\bot$ & $\bot$ & $\bot$ & 0 & 0 & 20400 & 20315 & 20050 & 20950 \\ \hline
      $\top$ & $\top$ & $\bot$ & $\bot$ & 0 & 0 & 19350 & 19265 & 18250 & 20650 \\ \hline
      $\top$ & $\top$ & $\top$ & $\bot$ & 0 & 0 & 21250 & 24250 & 21250 & 36250 \\ \hline
      $\top$ & $\top$ & $\top$ & $\top$ & 1 & 0 & 50800 & 49295 & 35550 & 51000
    \end{tabular}
    }
  \end{table}
  \end{frame}

\begin{frame}
  \frametitle{Opis algoritma}
    \begin{columns}[t]
      \column{0.5\textwidth}
      \begin{itemize}
      \item Parametri
        \begin{itemize}
          \item $\varrho$ - faktor zaglađivanja
          % \item $L$ - duljina rješenja
          \item $N$ - veličina uzorka
          \item $N_b$ - broj najboljih uzoraka
        \end{itemize}
      \end{itemize}
      \column{0.5\textwidth}
        \begin{itemize}
        \item Rad algoritma
        \begin{itemize}
          \item Početak
          \item Uzorkovanje
          \item Ocjenjivanje
          \item Ažuriranje
        \end{itemize}
        \end{itemize}
    \end{columns}
\end{frame}




\end{document}